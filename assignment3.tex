% Options for packages loaded elsewhere
\PassOptionsToPackage{unicode}{hyperref}
\PassOptionsToPackage{hyphens}{url}
%
\documentclass[
]{article}
\usepackage{lmodern}
\usepackage{amsmath}
\usepackage{ifxetex,ifluatex}
\ifnum 0\ifxetex 1\fi\ifluatex 1\fi=0 % if pdftex
  \usepackage[T1]{fontenc}
  \usepackage[utf8]{inputenc}
  \usepackage{textcomp} % provide euro and other symbols
  \usepackage{amssymb}
\else % if luatex or xetex
  \usepackage{unicode-math}
  \defaultfontfeatures{Scale=MatchLowercase}
  \defaultfontfeatures[\rmfamily]{Ligatures=TeX,Scale=1}
\fi
% Use upquote if available, for straight quotes in verbatim environments
\IfFileExists{upquote.sty}{\usepackage{upquote}}{}
\IfFileExists{microtype.sty}{% use microtype if available
  \usepackage[]{microtype}
  \UseMicrotypeSet[protrusion]{basicmath} % disable protrusion for tt fonts
}{}
\makeatletter
\@ifundefined{KOMAClassName}{% if non-KOMA class
  \IfFileExists{parskip.sty}{%
    \usepackage{parskip}
  }{% else
    \setlength{\parindent}{0pt}
    \setlength{\parskip}{6pt plus 2pt minus 1pt}}
}{% if KOMA class
  \KOMAoptions{parskip=half}}
\makeatother
\usepackage{xcolor}
\IfFileExists{xurl.sty}{\usepackage{xurl}}{} % add URL line breaks if available
\IfFileExists{bookmark.sty}{\usepackage{bookmark}}{\usepackage{hyperref}}
\hypersetup{
  pdftitle={Assignment3},
  pdfauthor={Minnie Ringland},
  hidelinks,
  pdfcreator={LaTeX via pandoc}}
\urlstyle{same} % disable monospaced font for URLs
\usepackage[margin=1in]{geometry}
\usepackage{color}
\usepackage{fancyvrb}
\newcommand{\VerbBar}{|}
\newcommand{\VERB}{\Verb[commandchars=\\\{\}]}
\DefineVerbatimEnvironment{Highlighting}{Verbatim}{commandchars=\\\{\}}
% Add ',fontsize=\small' for more characters per line
\usepackage{framed}
\definecolor{shadecolor}{RGB}{248,248,248}
\newenvironment{Shaded}{\begin{snugshade}}{\end{snugshade}}
\newcommand{\AlertTok}[1]{\textcolor[rgb]{0.94,0.16,0.16}{#1}}
\newcommand{\AnnotationTok}[1]{\textcolor[rgb]{0.56,0.35,0.01}{\textbf{\textit{#1}}}}
\newcommand{\AttributeTok}[1]{\textcolor[rgb]{0.77,0.63,0.00}{#1}}
\newcommand{\BaseNTok}[1]{\textcolor[rgb]{0.00,0.00,0.81}{#1}}
\newcommand{\BuiltInTok}[1]{#1}
\newcommand{\CharTok}[1]{\textcolor[rgb]{0.31,0.60,0.02}{#1}}
\newcommand{\CommentTok}[1]{\textcolor[rgb]{0.56,0.35,0.01}{\textit{#1}}}
\newcommand{\CommentVarTok}[1]{\textcolor[rgb]{0.56,0.35,0.01}{\textbf{\textit{#1}}}}
\newcommand{\ConstantTok}[1]{\textcolor[rgb]{0.00,0.00,0.00}{#1}}
\newcommand{\ControlFlowTok}[1]{\textcolor[rgb]{0.13,0.29,0.53}{\textbf{#1}}}
\newcommand{\DataTypeTok}[1]{\textcolor[rgb]{0.13,0.29,0.53}{#1}}
\newcommand{\DecValTok}[1]{\textcolor[rgb]{0.00,0.00,0.81}{#1}}
\newcommand{\DocumentationTok}[1]{\textcolor[rgb]{0.56,0.35,0.01}{\textbf{\textit{#1}}}}
\newcommand{\ErrorTok}[1]{\textcolor[rgb]{0.64,0.00,0.00}{\textbf{#1}}}
\newcommand{\ExtensionTok}[1]{#1}
\newcommand{\FloatTok}[1]{\textcolor[rgb]{0.00,0.00,0.81}{#1}}
\newcommand{\FunctionTok}[1]{\textcolor[rgb]{0.00,0.00,0.00}{#1}}
\newcommand{\ImportTok}[1]{#1}
\newcommand{\InformationTok}[1]{\textcolor[rgb]{0.56,0.35,0.01}{\textbf{\textit{#1}}}}
\newcommand{\KeywordTok}[1]{\textcolor[rgb]{0.13,0.29,0.53}{\textbf{#1}}}
\newcommand{\NormalTok}[1]{#1}
\newcommand{\OperatorTok}[1]{\textcolor[rgb]{0.81,0.36,0.00}{\textbf{#1}}}
\newcommand{\OtherTok}[1]{\textcolor[rgb]{0.56,0.35,0.01}{#1}}
\newcommand{\PreprocessorTok}[1]{\textcolor[rgb]{0.56,0.35,0.01}{\textit{#1}}}
\newcommand{\RegionMarkerTok}[1]{#1}
\newcommand{\SpecialCharTok}[1]{\textcolor[rgb]{0.00,0.00,0.00}{#1}}
\newcommand{\SpecialStringTok}[1]{\textcolor[rgb]{0.31,0.60,0.02}{#1}}
\newcommand{\StringTok}[1]{\textcolor[rgb]{0.31,0.60,0.02}{#1}}
\newcommand{\VariableTok}[1]{\textcolor[rgb]{0.00,0.00,0.00}{#1}}
\newcommand{\VerbatimStringTok}[1]{\textcolor[rgb]{0.31,0.60,0.02}{#1}}
\newcommand{\WarningTok}[1]{\textcolor[rgb]{0.56,0.35,0.01}{\textbf{\textit{#1}}}}
\usepackage{longtable,booktabs}
\usepackage{calc} % for calculating minipage widths
% Correct order of tables after \paragraph or \subparagraph
\usepackage{etoolbox}
\makeatletter
\patchcmd\longtable{\par}{\if@noskipsec\mbox{}\fi\par}{}{}
\makeatother
% Allow footnotes in longtable head/foot
\IfFileExists{footnotehyper.sty}{\usepackage{footnotehyper}}{\usepackage{footnote}}
\makesavenoteenv{longtable}
\usepackage{graphicx}
\makeatletter
\def\maxwidth{\ifdim\Gin@nat@width>\linewidth\linewidth\else\Gin@nat@width\fi}
\def\maxheight{\ifdim\Gin@nat@height>\textheight\textheight\else\Gin@nat@height\fi}
\makeatother
% Scale images if necessary, so that they will not overflow the page
% margins by default, and it is still possible to overwrite the defaults
% using explicit options in \includegraphics[width, height, ...]{}
\setkeys{Gin}{width=\maxwidth,height=\maxheight,keepaspectratio}
% Set default figure placement to htbp
\makeatletter
\def\fps@figure{htbp}
\makeatother
\setlength{\emergencystretch}{3em} % prevent overfull lines
\providecommand{\tightlist}{%
  \setlength{\itemsep}{0pt}\setlength{\parskip}{0pt}}
\setcounter{secnumdepth}{-\maxdimen} % remove section numbering
\ifluatex
  \usepackage{selnolig}  % disable illegal ligatures
\fi

\title{Assignment3}
\author{Minnie Ringland}
\date{2/1/2021}

\begin{document}
\maketitle

\hypertarget{part-1-exploring-the-data}{%
\subsection{Part 1: Exploring the
data}\label{part-1-exploring-the-data}}

For different countries, we have a time series of temperature (deg C)
and agricultural yield (kg/hectare).

What is the relationship between temperature and (log) yield?

\begin{Shaded}
\begin{Highlighting}[]
\NormalTok{panel }\OtherTok{\textless{}{-}} \FunctionTok{read.csv}\NormalTok{(}\StringTok{"country\_year\_panel.csv"}\NormalTok{, }\AttributeTok{header=}\ConstantTok{TRUE}\NormalTok{)}

\CommentTok{\# Overlapping plots}
\FunctionTok{plot}\NormalTok{(panel}\SpecialCharTok{$}\NormalTok{temp, panel}\SpecialCharTok{$}\NormalTok{l\_yield)}
\FunctionTok{abline}\NormalTok{(}\FunctionTok{lm}\NormalTok{(panel}\SpecialCharTok{$}\NormalTok{l\_yield}\SpecialCharTok{\textasciitilde{}}\NormalTok{panel}\SpecialCharTok{$}\NormalTok{temp), }\AttributeTok{lwd=}\DecValTok{3}\NormalTok{, }\AttributeTok{col=}\StringTok{"blue"}\NormalTok{)}
\FunctionTok{lines}\NormalTok{(}\FunctionTok{lowess}\NormalTok{(panel}\SpecialCharTok{$}\NormalTok{temp, panel}\SpecialCharTok{$}\NormalTok{l\_yield, }\AttributeTok{f=}\NormalTok{.}\DecValTok{1}\NormalTok{, }\AttributeTok{iter=}\DecValTok{0}\NormalTok{), }\AttributeTok{lwd=}\DecValTok{3}\NormalTok{, }\AttributeTok{col=}\StringTok{"red"}\NormalTok{)}
\end{Highlighting}
\end{Shaded}

\includegraphics{assignment3_files/figure-latex/unnamed-chunk-1-1.pdf}
Black circles = observations\\
Blue line = linear regression\\
Red line = non-parametric local linear regression

The spread of the data points and the variation in the red line, in
addition to the low temperature cluster, show that there may be patterns
in the data that a linear approximation will leave out.

\begin{center}\rule{0.5\linewidth}{0.5pt}\end{center}

\hypertarget{part-2-cross-sectional-vs.-within-models-all-countries}{%
\subsection{Part 2: Cross-sectional vs.~within models: all
countries}\label{part-2-cross-sectional-vs.-within-models-all-countries}}

\hypertarget{average-temperature-climate}{%
\subsubsection{Average temperature (\textasciitilde{}
climate)}\label{average-temperature-climate}}

We'll average all observations for a country across all years, and
generate a single mean set of values for each country.

\begin{Shaded}
\begin{Highlighting}[]
\CommentTok{\# "Collapse" into country means}
\NormalTok{cross }\OtherTok{\textless{}{-}} \FunctionTok{summaryBy}\NormalTok{(l\_yield}\SpecialCharTok{+}\NormalTok{temp }\SpecialCharTok{\textasciitilde{}}\NormalTok{ cID, }\AttributeTok{FUN=}\FunctionTok{c}\NormalTok{(mean), }\AttributeTok{data=}\NormalTok{panel)}

\CommentTok{\# Then plot again}
\FunctionTok{plot}\NormalTok{(cross}\SpecialCharTok{$}\NormalTok{temp.mean, cross}\SpecialCharTok{$}\NormalTok{l\_yield.mean)}
\FunctionTok{abline}\NormalTok{(}\FunctionTok{lm}\NormalTok{(cross}\SpecialCharTok{$}\NormalTok{l\_yield.mean}\SpecialCharTok{\textasciitilde{}}\NormalTok{cross}\SpecialCharTok{$}\NormalTok{temp.mean), }\AttributeTok{lwd=}\DecValTok{3}\NormalTok{, }\AttributeTok{col=}\StringTok{"blue"}\NormalTok{)}
\FunctionTok{lines}\NormalTok{(}\FunctionTok{lowess}\NormalTok{(cross}\SpecialCharTok{$}\NormalTok{temp.mean, cross}\SpecialCharTok{$}\NormalTok{l\_yield.mean, }\AttributeTok{f=}\NormalTok{.}\DecValTok{1}\NormalTok{, }\AttributeTok{iter=}\DecValTok{0}\NormalTok{), }\AttributeTok{lwd=}\DecValTok{3}\NormalTok{, }\AttributeTok{col=}\StringTok{"red"}\NormalTok{)}
\end{Highlighting}
\end{Shaded}

\includegraphics{assignment3_files/figure-latex/unnamed-chunk-2-1.pdf}

Now, the spread of data is reduced and the linear approximation seems
more reasonable - the single low temperature observation appears to be
an outlier. If we move forward with the assumption of a linear
relationship, we can calculate the approximate effect on temperature on
yield:

\begin{Shaded}
\begin{Highlighting}[]
\CommentTok{\# Calculate coefficient on temperature using the linear model}
\NormalTok{linear\_temp\_coeff }\OtherTok{\textless{}{-}} \FunctionTok{lm}\NormalTok{(cross}\SpecialCharTok{$}\NormalTok{l\_yield.mean}\SpecialCharTok{\textasciitilde{}}\NormalTok{cross}\SpecialCharTok{$}\NormalTok{temp.mean)}\SpecialCharTok{$}\NormalTok{coefficients[}\DecValTok{2}\NormalTok{]}
\end{Highlighting}
\end{Shaded}

Temperature coefficient using the linear model = \emph{-0.0549831}\\
This means we'd expect a 5\% reduction (negative relationship) in yield
from a 1 degree increase in temperature.

\hypertarget{cross-sectional-estimate-with-the-within-or-fixed-effects-estimate}{%
\subsubsection{Cross-sectional estimate with the ``within'' or fixed
effects
estimate}\label{cross-sectional-estimate-with-the-within-or-fixed-effects-estimate}}

Now we will try a model that enables a different type comparison by
grouping (by country) and then measuring within-group deviation from the
country mean:

\begin{Shaded}
\begin{Highlighting}[]
\CommentTok{\# Running a fixed effect regression}
\NormalTok{fixed\_effects\_coeff }\OtherTok{\textless{}{-}} \FunctionTok{plm}\NormalTok{(l\_yield }\SpecialCharTok{\textasciitilde{}}\NormalTok{ temp, }\AttributeTok{data=}\NormalTok{panel, }\AttributeTok{index=}\FunctionTok{c}\NormalTok{(}\StringTok{"cID"}\NormalTok{, }\StringTok{"year"}\NormalTok{), }\AttributeTok{model=}\StringTok{"within"}\NormalTok{)}\SpecialCharTok{$}\NormalTok{coefficients}
\end{Highlighting}
\end{Shaded}

Temperature coefficient using the fixed effect regression =
\emph{-0.0546861}\\
The expected impact is similar to our previous model - about 5\%
reduction in yield per degree temperature increase.

These two models have different advantages - the cross-sectional model
should capture adaptation to temperature change because we average
across time, and the fixed effects model should account for variation in
temperature between countries. Since these models agree, we can surmise
that most countries have been able to adapt their agricultural
techniques to make up for any temperature variations they experienced,
and that the limit of the climate effect may be \textasciitilde5\% yield
loss.

\begin{center}\rule{0.5\linewidth}{0.5pt}\end{center}

\hypertarget{part-3-cross-sectional-vs.-within-models-continents}{%
\subsection{Part 3: Cross-sectional vs.~within models:
continents}\label{part-3-cross-sectional-vs.-within-models-continents}}

We have reason to believe that climate and weather effects are
heterogeneous - they may be different for different parts of the world.
So we'll redo our regressions for each continent to try to isolate
geographic patterns.

\begin{Shaded}
\begin{Highlighting}[]
\NormalTok{continents }\OtherTok{\textless{}{-}} \FunctionTok{c}\NormalTok{(}\StringTok{"AF"}\NormalTok{,}\StringTok{"AS"}\NormalTok{,}\StringTok{"EU"}\NormalTok{, }\StringTok{"OC"}\NormalTok{,}\StringTok{"SA"}\NormalTok{,}\StringTok{"NM"}\NormalTok{)}

\NormalTok{part3 }\OtherTok{\textless{}{-}} \FunctionTok{data.frame}\NormalTok{(}\StringTok{"CrossSectional"} \OtherTok{=} \DecValTok{1}\SpecialCharTok{:}\DecValTok{6}\NormalTok{, }\StringTok{"FixedEffects"} \OtherTok{=} \DecValTok{1}\SpecialCharTok{:}\DecValTok{6}\NormalTok{, }\AttributeTok{row.names =}\NormalTok{ continents)}

\ControlFlowTok{for}\NormalTok{ (i }\ControlFlowTok{in}\NormalTok{ continents) \{}
\NormalTok{  continent\_panel }\OtherTok{\textless{}{-}} \FunctionTok{subset}\NormalTok{(panel, panel}\SpecialCharTok{$}\NormalTok{continent}\SpecialCharTok{==}\NormalTok{i, }\AttributeTok{select=}\FunctionTok{c}\NormalTok{(cID, year, l\_yield, temp, continent))}
\NormalTok{  continent\_cross }\OtherTok{\textless{}{-}} \FunctionTok{summaryBy}\NormalTok{(l\_yield}\SpecialCharTok{+}\NormalTok{temp }\SpecialCharTok{\textasciitilde{}}\NormalTok{ cID, }\AttributeTok{FUN=}\FunctionTok{c}\NormalTok{(mean), }\AttributeTok{data=}\NormalTok{continent\_panel)}
\NormalTok{  part3[i,}\DecValTok{1}\NormalTok{] }\OtherTok{\textless{}{-}} \FunctionTok{lm}\NormalTok{(continent\_cross}\SpecialCharTok{$}\NormalTok{l\_yield.mean}\SpecialCharTok{\textasciitilde{}}\NormalTok{continent\_cross}\SpecialCharTok{$}\NormalTok{temp.mean)}\SpecialCharTok{$}\NormalTok{coefficients[}\DecValTok{2}\NormalTok{]}
\NormalTok{  part3[i,}\DecValTok{2}\NormalTok{] }\OtherTok{\textless{}{-}} \FunctionTok{plm}\NormalTok{(l\_yield }\SpecialCharTok{\textasciitilde{}}\NormalTok{ temp, }\AttributeTok{data=}\NormalTok{continent\_panel, }\AttributeTok{index=}\FunctionTok{c}\NormalTok{(}\StringTok{"cID"}\NormalTok{, }\StringTok{"year"}\NormalTok{), }\AttributeTok{model=}\StringTok{"within"}\NormalTok{)}\SpecialCharTok{$}\NormalTok{coefficients}
\NormalTok{\}}

\FunctionTok{kable}\NormalTok{(part3, }\AttributeTok{caption =} \StringTok{"Coefficient Comparison Across Continents"}\NormalTok{)}
\end{Highlighting}
\end{Shaded}

\begin{longtable}[]{@{}lrr@{}}
\caption{Coefficient Comparison Across Continents}\tabularnewline
\toprule
& CrossSectional & FixedEffects\tabularnewline
\midrule
\endfirsthead
\toprule
& CrossSectional & FixedEffects\tabularnewline
\midrule
\endhead
AF & -0.0261016 & -0.1181200\tabularnewline
AS & -0.0090463 & -0.0451354\tabularnewline
EU & -0.0610025 & -0.0168930\tabularnewline
OC & -0.0844075 & -0.1989222\tabularnewline
SA & -0.0091302 & -0.0816297\tabularnewline
NM & -0.0297575 & 0.0067167\tabularnewline
\bottomrule
\end{longtable}

\begin{Shaded}
\begin{Highlighting}[]
\FunctionTok{ggplot}\NormalTok{(}\AttributeTok{data=}\NormalTok{part3)}\SpecialCharTok{+}
  \FunctionTok{geom\_col}\NormalTok{(}\FunctionTok{aes}\NormalTok{(}\AttributeTok{x=}\FunctionTok{row.names}\NormalTok{(part3), }\AttributeTok{y=}\NormalTok{CrossSectional), }\AttributeTok{fill=}\StringTok{"blue"}\NormalTok{) }\SpecialCharTok{+}
  \FunctionTok{geom\_col}\NormalTok{(}\FunctionTok{aes}\NormalTok{(}\AttributeTok{x=}\FunctionTok{row.names}\NormalTok{(part3), }\AttributeTok{y=}\NormalTok{FixedEffects), }\AttributeTok{fill=}\StringTok{"green"}\NormalTok{) }\SpecialCharTok{+}
  \FunctionTok{labs}\NormalTok{(}\AttributeTok{x =} \StringTok{"Country"}\NormalTok{, }\AttributeTok{y =} \StringTok{"Temp Coefficient"}\NormalTok{)}
\end{Highlighting}
\end{Shaded}

\includegraphics{assignment3_files/figure-latex/unnamed-chunk-5-1.pdf}

Now we see that the coefficients produced by the two models do not agree
when we subset by continent.

Are the coefficients between cross sectional and fixed effects panel
regressions different for each continent. What does this tell us about
adaptation in each continent. Does it make sense? Where does it not make
sense and what are possible explanations?

\begin{center}\rule{0.5\linewidth}{0.5pt}\end{center}

\hypertarget{bonus}{%
\subsection{Bonus}\label{bonus}}

\begin{Shaded}
\begin{Highlighting}[]
\CommentTok{\# Manual fixed effects regression}
\CommentTok{\# First, de{-}mean:}

\CommentTok{\#manual \textless{}{-} panel \%\textgreater{}\% }
\CommentTok{\#  mutate(l\_yield.mean = 0)}


\CommentTok{\#  mutate(l\_yield.demean = l\_yield.mean {-} l\_yield\_mean) \%\textgreater{}\% }
\CommentTok{\#  mutate(temp.demean = temp.mean {-} temp\_mean)}

\CommentTok{\#manual\_coeff \textless{}{-} lm(manual$l\_yield.demean\textasciitilde{}manual$temp.demean)$coefficients[1]}
\CommentTok{\#Temperature coefficient using manually coded fixed effects regression = *\textasciigrave{}r manual\_coeff\textasciigrave{}*  }
\end{Highlighting}
\end{Shaded}

Temperature coefficient using R's fixed effect regression =
\emph{-0.0546861}

\begin{Shaded}
\begin{Highlighting}[]
\CommentTok{\# Then plot again}
\CommentTok{\#plot(manual$temp.demean, manual$l\_yield.demean)}
\CommentTok{\#abline(lm(manual$l\_yield.demean\textasciitilde{}manual$temp.demean), lwd=3, col="blue")}
\CommentTok{\#lines(lowess(manual$temp.demean, manual$l\_yield.demean, f=.1, iter=0), lwd=3, col="red")}
\end{Highlighting}
\end{Shaded}


\end{document}
